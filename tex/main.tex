\documentclass[11pt, class=article, crop=false]{standalone}
\usepackage[subpreambles=true]{standalone}
\usepackage[T1]{fontenc} % for font setting
\usepackage{tgtermes} % for font setting
\usepackage{import,
            graphicx,
            parskip,
            url,
            amsmath,
            wrapfig,
            fancyhdr,
            soul,
            tabularx}

% side caption figure
\usepackage{sidecap}
\sidecaptionvpos{figure}{t}

% for special characters in bibliography            
\usepackage[utf8]{inputenc}
\usepackage[T1]{fontenc}

% citation setup
\usepackage[euler]{textgreek}
\usepackage[sort&compress]{natbib}
\setcitestyle{square}
\setcitestyle{comma}
\bibliographystyle{bibstyle}

% caption setup
\usepackage[font={f, small}, labelfont={bf, small}]{caption}
           
% color box
\usepackage[most]{tcolorbox}
\tcbuselibrary{breakable}

% margin
\usepackage[top=2.54cm, bottom=2.54cm, left=2.54cm, right=2.54cm]{geometry}%set margin

% \pagenumbering{gobble}

\begin{document}

\section{Maintext}

The vertical structure of food webs, often quantified as food chain length (FCL), has important implications for trophic dynamics, energy flow, and contaminant concentrations in top predators that humans consume.
Elton first documented its variation in natural systems in the early XXX.
Since then, controls of FCL have been a central topic in ecology with a diverse array of hypotheses spawned over the past century.

Hypotheses regarding controls of food chain length can be organized into three broad categories: resource availability, disturbance (or dynamic stability), and ecosystem size.
Ecological theory predicts that FCL should increase with resource availability and environmental stability because such conditions may increase the persistence of constituent species, particularly those of higher trophic levels.
A sizable amount of research has reported results contradicting these theoretical predictions.
In contrast, a meta-analysis provided consistent support for the ecosystem size hypothesis, in which larger ecosystems are predicted to support longer food chains.
Recent theory suggests multiple mechanisms behind the positive association between FCL and ecosystem size, such as weakened intraguild predation and increased colonization.
This multifaceted nature may underlie the consistency of the positive relationship.

XXX proposed the resource availability with the longest history of research.
FCL is ultimately controlled by basal resources because inefficient energy transfer through food webs limits energy available to higher trophic levels.


Second, the disturbance hypothesis (or the dynamical stability hypothesis) emerges from classic theoretical predictions: longer food chains are less resilient to environmental perturbations (i.e., the dynamical constraint) and unlikely to persist in habitats with frequent or severe disturbance.
Lastly, the ecosystem size hypothesis predicts that larger ecosystems, or larger networks of local habitats coupled via dispersal, should support longer food chains through a suite of mechanisms, including enhanced colonization rates and weakened intraguild predation.

\section{Results and Discussion}
\section{Methods}


\end{document}