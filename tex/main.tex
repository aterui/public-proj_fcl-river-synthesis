\documentclass[11pt, class=article, crop=false]{standalone}
\usepackage[subpreambles=true]{standalone}
\usepackage[T1]{fontenc} % for font setting
\usepackage{tgtermes} % for font setting
\usepackage{import,
            graphicx,
            parskip,
            url,
            amsmath,
            wrapfig,
            fancyhdr,
            soul,
            tabularx}

% side caption figure
\usepackage{sidecap}
\sidecaptionvpos{figure}{t}

% for special characters in bibliography            
\usepackage[utf8]{inputenc}
\usepackage[T1]{fontenc}

% citation setup
\usepackage[euler]{textgreek}
\usepackage[sort&compress]{natbib}
\setcitestyle{square}
\setcitestyle{comma}
\bibliographystyle{bibstyle}

% caption setup
\usepackage[font={f, small}, labelfont={bf, small}]{caption}
           
% color box
\usepackage[most]{tcolorbox}
\tcbuselibrary{breakable}

% margin
\usepackage[top=2.54cm, bottom=2.54cm, left=2.54cm, right=2.54cm]{geometry}%set margin

% \pagenumbering{gobble}

\begin{document}

\section{Maintext}

The vertical structure of food webs, often quantified as food chain length (FCL), has important implications for trophic dynamics, energy flow, and biomagnification of environmental pollutants.
Elton first documented its variation in natural systems in the early XXX.
Since then, controls of FCL have been a central topic in ecology with a diverse array of hypotheses spawned over the past century.

Hypothetical controls of FCL can be organized into three broad categories: resource availability, disturbance (or dynamic stability), and ecosystem size.
FCL should increase with resource availability and environmental stability because such conditions may increase the persistence of constituent species, particularly those of higher trophic levels.
A sizable amount of research, however, has reported results contradicting these theoretical predictions, implying the complex interactions between internal food web structure and environmental drivers.
In contrast, several meta-analyses revealed consistent support for the ecosystem size hypothesis, in which larger ecosystems are predicted to support longer food chains.
Recent theory suggests multiple mechanisms, such as weakened intraguild predation and enhanced colonization, which are hypothesized to underlie the consistency of the positive association between FCL and ecosystem size.


\section{Results and Discussion}
\section{Methods}


\end{document}