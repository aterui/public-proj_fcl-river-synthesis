% We derive the expected value of the upstream river length, $\mathbb{E}(u)$, at a given habitat patch within a branching network characterized by total river length $L$ and branching rate $\lambda_b$.
% Our objective is to express $\mathbb{E}(u)$ as a function of $L$ and $\lambda_b$, considering the probabilistic nature of river bifurcation.
% In this section, we use the following key terms: \textbf{link} and \textbf{magnitude}.
% The term \textbf{link} refers to a river segment from one confluence to another or to a terminal point of the river (i.e., the upstream origin or the river mouth).
% The term \textbf{magnitude} refers to the number of exterior source links (links containing upstream terminals) at a given position within the network. The magnitude is one at an exterior source link.
% Table \ref{tab:key-symbol} lists the key mathematical symbols used in this section.


% \begin{table}
%     \centering
%     \caption{Key symbols for the derivation of expected upstream river length}
%     \begin{tabularx}{\textwidth}{ll}
%         Symbol & Description\\
%         \hline
%         $u$ & upstream river length at a given habitat patch.\\
%         $L$ & Total river length.\\
%         $\lambda_b$ & Branching rate defining the length distribution of individual links.\\
%         $l$ & Individual link length. Assumed to follow $l \sim \mbox{Exp}(\lambda_b)$.\\
%         $B$ & Number of links ($B = Z + 1$).\\
%         $Z$ & Number of links minus one ($Z = B - 1$).\\
%         \hline
%     \end{tabularx}
%     \label{tab:key-symbol}
% \end{table}

% In our theoretical framework, we assume that the length of individual link $s$, denoted as $l_s$, follows an exponential distribution $l_s \sim \mbox{Exp}(\lambda_b)$, but is conditional on the constraint $\sum_s l_s = L$.
% The length $l_s$ can be interpreted as the ``waiting time'' to a branching event (or termination events for upstream sources).
% According to statistical theory, the number of such events within a given duration follows a Poisson distribution.
% In our case, this translates into counting the number of branching events $Z$ within the ``time'' frame $L$, excluding the final branching event (since link termination occurs at $L$ regardless).
% Thus, $Z \sim \mbox{Pois}(\lambda_b L)$.
% The total number of links $B$ equals $Z + 1$, where the addition of one accounts for the final event of link termination at length $L$.
% We use these assumptions in the following derivation.

% The expected upstream river length, $\mathbb{E}(u)$, can be decomposed as $\mathbb{E}(u) = \mathbb{E}(u') + \mathbb{E}(l')$, where $u'$ and $l'$ represent the summed length of upstream links (excluding the link containing the habitat patch) and the distance from the upstream link end to the habitat patch, respectively.
% In the following, we first derive the expected values of $u'$ and $l'$ when the random variables $Z$ and $B$ are given as $Z = z$ ($z \in \{\text{even integer}\}$, $z \ge 0$) and $B = b = z + 1$ ($b \in \{\text{odd integer}\}$, $b \ge 1$).
% We clarify this conditional assumption by denoting $\cdot ~|~ z$ in the equations.
% We account for the randomness of $Z$ as the final step of the derivation.

% The summed length of upstream links $u'$ is the product of the mean length of upstream links $\hat{l}$ and the number of upstream links $n$.
% If $\lambda_b L \rightarrow \infty$ and each link length follows the \textit{i.i.d.} assumption, the expected value of mean length $\hat{l}$ ($= \sum_s^n l_s n^{-1}$) can be approximated as $\lambda_b^{-1}$ (the expected value of an exponential distribution) since one may ignore the length constraint imposed by the condition $\sum_s l_s = L$.
% If $\lambda_b L$ is not sufficiently large, however, we need to account for this constraint.
% With the known sum $L$, the fraction of single link length $\psi_s$ (= $l_s L^{-1}$) follows a Beta distribution as $\psi_s ~|~ z \sim \mbox{Beta}(1, z)$ (\textbf{Lemma 1}), whose expected value is $\mathbb{E}(\psi~|~z) = (z + 1)^{-1}$.
% Thus, we obtain an intuitive solution for the expectation of mean length as:

% \begin{align}
%     \begin{split}
%     \mathbb{E}(\hat{l}~|~z) &= L \cdot \mathbb{E}(\psi~|~z)\\
%                           &= L(z + 1)^{-1}\\
%                           &= Lb^{-1}.
%     \end{split}
% \end{align}

% The expected value of this quantity, $\mathbb{E}(LB^{-1})$, approaches $\lambda_b^{-1}$ as $\lambda_b L \rightarrow \infty$ since $\mathbb{E}(B) = \lambda_b L + 1$.

% The number of upstream links $n$ requires the consideration of its probability distribution since $n$ depends on the location in a network.
% Here, our approach leverages the known properties of Shreve's topologically distinct channel networks (TDCNs), a classical framework of bifurcating river networks upon which our theory builds.
% Specifically, the probability of drawing a link with magnitude $m$ at random from a population of $M$-magnitude TDCNs, $\omega(m; M)$, is described as:

% \begin{equation}
%     \omega(m; M) = \frac{1}{2m - 1} \binom{2m}{m} \binom{2(M - m)}{M - m} \binom{2M}{M}^{-1}.
% \end{equation}

% The notation $\binom{\cdot}{\cdot}$ and $\binom{\cdot}{\cdot}^{-1}$ denote a binomial coefficient and its inverse.
% The number of upstream links $n$ has a linear relationship with link magnitude $m$ as $n = 2m - 2$ due to the bifurcating nature of TDCNs, leading to the expected number of upstream links $\mathbb{E}(n~|~z)$ as:

% \begin{align}
%     \begin{split}
%     \mathbb{E}(n~|~z) &= \sum_{m=1}^{M} (2m - 2) \omega(m; M)\\
%                     &= 2 \sum_{m=1}^{M} m~\omega(m; M) - 2\\
%                     &= 2 \cdot 4^{M-1} \binom{2M-1}{M}^{-1} - 2\\
%                     &= 2^{2M-1} \binom{2M-1}{M}^{-1} - 2.
%     \end{split}
%     \label{eq:n-hat}
% \end{align}

% Note that $M$ is a function of $z$.
% TDCNs satisfy the relationship $b = 2M - 1$ because two source links must join to form a single downstream link.
% Using $M = (b + 1) / 2 = (z + 2) / 2$, the expected value $\mathbb{E}(n~|~z)$ can be rewritten as a function of $z$:

% \begin{equation}
%     \mathbb{E}(n~|~z) = \hat{n}(z) = 
%     2^{z + 1} \binom{z + 1}{\frac{z + 2}{2}}^{-1} - 2.
% \end{equation}

% In sum, we obtain $\mathbb{E}(u' ~|~ z)$ as:

% \begin{align}
%     \begin{split}
%         \mathbb{E}(u' ~|~ z) &= \mathbb{E}(\hat{l}n ~|~ z)\\
%                            &= \mathbb{E}(\hat{l} ~|~ z) \cdot \mathbb{E}(n ~|~ z)\\
%                            &= \frac{L \hat{n}(z)}{z + 1}.
%     \end{split}
% \end{align}

% We used $\mathbb{E}(\hat{l}n ~|~ z) = \mathbb{E}(\hat{l} ~|~ z) \cdot \mathbb{E}(n ~|~ z)$ in light of the lack of covariance between $\hat{l}$ and $n$.

% The derivation of $\mathbb{E}(l'~|~z)$ is straightforward.
% We assume that habitat patches are randomly distributed along the river, following a Poisson process.
% As such, the distance to habitat patch $i$ in link $s$, $l'_{s(i)}$, follows a uniform distribution as $l'_{s(i)}~|~z \sim \mbox{Unif}(0, l_s)$ with $\mathbb{E}(l'_{s(i)}~|~z) = 2^{-1} l_s$.
% Taking the expectation across links yields:

% \begin{align}
%     \begin{split}
%         \mathbb{E}(l'~|~z) &= \mathbb{E}(\mathbb{E}(l'_{s(i)}~|~z))\\
%                      &= \frac{\mathbb{E}(l_s~|~z)}{2}\\
%                      &= \frac{L}{2(z + 1)}.
%     \end{split}
% \end{align}

% Summing $\mathbb{E}(u' ~|~ z)$ and $\mathbb{E}(l'~|~z)$ for a network with a given value of $z$ yields:

% \begin{align}
%     \begin{split}
%         \mathbb{E}(u ~|~ z) &= \mathbb{E}(u' ~|~ z) + \mathbb{E}(l'~|~z)\\
%                           &= \frac{L}{z+1} \left[ \hat{n}(z) + \frac{1}{2} \right].
%     \end{split}
%     \label{eq:cu}
% \end{align}

% Lastly, we account for the probabilistic nature of the random variable $Z$ by summing $z$ out from Equation \ref{eq:cu}.
% In a bifurcating network, $Z$ must be even so that the total number of links $B$ is odd.
% As such, we must consider a truncated Poisson distribution for $Z$ with zero probabilities for odd integers.
% The probability mass function (PMF) of the truncated Poisson distribution, denoted as $f'_{\text{pois}}(z; L, \lambda_b)$, is defined as:

% \begin{align}
%     \begin{split}
%     f'_{\text{pois}}(z; L, \lambda_b) 
%     &= \frac{f_{\text{pois}}(z; L, \lambda_b)}{\Pr(Z = \text{even})} \cdot \frac{1 + (-1)^{z}}{2}\\
%     &= f_{\text{pois}}(z; L, \lambda_b) \cdot \frac{1 + (-1)^{z}}{1 + e^{-2\lambda_b L}}
%     \end{split}
%     \label{eq:tpois}
% \end{align}

% where $f_{\text{pois}}(z; L, \lambda_b)$ is the PMF of a Poisson distribution:

% \begin{equation}
%     f_{\text{pois}}(z; L, \lambda_b) = \Pr(Z = z) = \frac{(\lambda_b L)^{z} e^{-\lambda_b L}}{z!},
% \end{equation}

% and $\Pr(Z = \text{even}) = 2^{-1}(1 + e^{- 2 \lambda_b L})$ (\textbf{Lemma 3}). 
% In Equation \ref{eq:tpois}, $f_{\text{pois}}(z; L, \lambda_b)$ is divided by the probability of $Z$ taking even integers $\Pr(Z = \text{even})$ so that $\sum_{z=0}^{\infty} f'_{\text{pois}}(z; L, \lambda_b) = 1$.
% Following the definition of expected value, we obtain $\mathbb{E}(u)$ as a function of $L$ and $\lambda_b$:

% \begin{align}
%     \begin{split}
%         \mathbb{E}(u) = \hat{u}(L, \lambda_b) 
%                     &= \sum_{z = 0}^{\infty} \left[ \frac{L}{z + 1} \left(\hat{n}(z) + \frac{1}{2}\right) f'_{\text{pois}}(z; L, \lambda_b) \right]\\
%                     &= L \sum_{z = 0}^{\infty} \left[ \frac{\hat{n}(z)}{z + 1} f'_{\text{pois}}(z; L, \lambda_b)\right] + 
%                     \frac{1 - e^{-2 \lambda_b L}}{2 \lambda_b (1 + e^{-2 \lambda_b L})}.
%     \end{split}
% \end{align}

% $\Pr(Z = \text{even})$ is given as:
% \begin{align}
%     \begin{split}
%         \Pr(Z = \text{even}) &= \sum_{z = 0}^{\infty} \left[ \frac{(\lambda_b L)^{z} e^{-\lambda_b L}}{z!} \times \frac{1 + (-1)^{z}}{2} \right]\\
%                          &= \frac{e^{-\lambda_b L}}{2} \left[ \sum_{z = 0}^{\infty} \frac{(\lambda_b L)^{z}}{z!} + \sum_{z = 0}^{\infty} \frac{(-\lambda_b L)^{z}}{z!}\right]\\
%                          &= \frac{e^{-\lambda_b L}}{2} (e^{\lambda_b L} + e^{-\lambda_b L})\\
%                          &= \frac{1 + e^{- 2 \lambda_b L}}{2}    
%     \end{split}
% \end{align}