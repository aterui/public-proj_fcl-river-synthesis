\begin{table}[ht]
\centering
\caption{List of publications used for our meta-analysis.
             \label{tab:meta-list}} 
\begingroup\small
\begin{tabularx}{\textwidth}{rl}
  \hline
ID & Publication \\ 
  \hline
  1 & Wilkinson, C.L. *et al*., Forest conversion to oil palm compresses food chain length in tropical streams. *Ecology* 102: e03199 \\ 
    2 & Jackson, M.C. *et al*., Food web properties vary with climate and land use in South African streams. *Functional Ecology* 34: 1653-1665 \\ 
    3 & Sroczynska, K. *et al*., Food web structure of three Mediterranean stream reaches along a gradient of anthropogenic impact. *Hydrobiologia* 847: 2357-2375 \\ 
    4 & Kelson, S.J. *et al*., Partial migration alters population ecology and food chain length: evidence from a salmonid fish. *Ecosphere* 11: e03044 \\ 
    5 & Boddy, N.C. *et al*., Big impacts from small abstractions: The effects of surface water abstraction on freshwater fish assemblages. *Aquatic Conservation: Marine and Freshwater Ecosystems* 30: 159-172 \\ 
    6 & Pearce, J.L. *et al*., Unrestricted ramping rates and long-term trends in the food web metrics of a boreal river. *River Research and Applications* 35: 1575-1589 \\ 
    7 & Negishi, J.N. *et al*., High resilience of aquatic community to a 100-year flood in a gravel-bed river. *Landscape and Ecological Engineering* 15: 143-154 \\ 
    8 & Pease, A.A. *et al*., Trophic structure of fish assemblages varies across a Mesoamerican river network with contrasting climate and flow conditions. *Food Webs* 18: e00113 \\ 
    9 & Sullivan, S.M.P. *et al*., Artificial lighting at night alters aquatic-riparian invertebrate food webs. *Ecological Applications* 29: e01821 \\ 
   10 & Fraley, K.M. *et al*., Responsiveness of fish mass-abundance relationships and trophic metrics to flood disturbance, stream size, land cover and predator taxa presence in headwater streams. *Ecology of Freshwater Fish* 27: 999-1014 \\ 
   11 & Gonzalez-Bergonzoni, I. *et al*., Riparian forest modifies fuelling sources for stream food webs but not food-chain length in lowland streams of Denmark. *Hydrobiologia* 805: 291-310 \\ 
   12 & Kaymak, N. *et al*., Spatial and temporal variation in food web structure of an impounded river in Anatolia. *Marine and Freshwater Research* 69: 1453-1471 \\ 
   13 & Ceneviva-Bastos, M. *et al*., Responses of aquatic food webs to the addition of structural complexity and basal resource diversity in degraded Neotropical streams. *Austral Ecology* 42: 908-919 \\ 
   14 & Taylor, G.C. *et al*., Comparing the fish assemblages and food-web structures of large floodplain rivers. *Freshwater Biology* 62: 1891-1907 \\ 
   15 & Majdi, N. and Traunspurger, W.Leaf fall affects the isotopic niches of meiofauna and macrofauna in a stream food web. *Food Webs* 10: 5-14 \\ 
   16 & Kautza, A. and Sullivan, S.M.P.Anthropogenic and natural determinants of fish food-chain length in a midsize river system. *Freshwater Science* 35: 895-908 \\ 
   17 & Ruhi, A. *et al*., Flow regulation increases food-chain length through omnivory mechanisms in a Mediterranean river network. *Freshwater Biology* 61: 1536-1549 \\ 
   18 & Jardine, T.D., A top predator forages low on species-rich tropical food chains. *Freshwater Science* 35: 666-675 \\ 
   19 & Hette-Tronquart, N. *et al*., Variability of water temperature may influence food-chain length in temperate streams. *Hydrobiologia* 718: 159-172 \\ 
   20 & Riva-Murray, K. *et al*., Spatial patterns of mercury in macroinvertebrates and fishes from streams of two contrasting forested landscapes in the eastern United States. *Ecotoxicology* 20: 1530-1542 \\ 
   21 & Pilger, T. J. *et al*., Diet and trophic niche overlap of native and nonnative fishes in the Gila River, USA: Implications for native fish conservation. *Ecology of Freshwater Fish* 19: 300-321 \\ 
   22 & Singer, G.A. and Battin, T.J.Anthropogenic subsidies alter stream consumer-resource stoichiometry, biodiversity, and food chains. *Ecological Applications* 17: 376-389 \\ 
   23 & Orr, P.L. *et al*., Food chain transfer of selenium in lentic and lotic habitats of a western Canadian watershed. *Ecotoxicology and Environmental Safety* 63: 175-188 \\ 
   24 & Lake, J.L. *et al*., Stable nitrogen isotopes as indicators of anthropogenic activities in small freshwater systems. *Canadian Journal of Fisheries and Aquatic Sciences* 58: 870-878 \\ 
   25 & Angradi T.R., Trophic linkages in the lower Colorado River: Multiple stable isotope evidence. *Journal of the North American Benthological Society* 13: 479-495 \\ 
   26 & Bunn, S.E. *et al*., Contributions of sugar cane and invasive pasture grass to the aquatic food web of a tropical lowland stream. *Marine and Freshwater Research* 48: 173-179 \\ 
   27 & Perry, R.W. *et al*., Effects of disturbance on contribution of energy sources to growth of juvenile chinook salmon (Oncorhynchus tshawytscha) in boreal streams. *Canadian Journal of Fisheries and Aquatic Sciences* 60: 390-400 \\ 
   28 & Doucett, R.R. *et al*., Stable isotope analysis of nutrient pathways leading to Atlantic salmon. *Canadian Journal of Fisheries and Aquatic Sciences* 53: 2058-2066 \\ 
   29 & Charles, K. *et al*., Estimating the contribution of sympatric anadromous and freshwater resident brown trout to juvenile production. *Marine and Freshwater Research* 55: 185-191 \\ 
   30 & March, J.G. and Pringle, C.M.Food web structure and basal resource utilization along a tropical island stream continuum, Puerto Rico. *Biotropica* 35: 84-93 \\ 
   31 & Steffy, L. and Kilham, S.S.Elevated delta N-15 in stream biota in areas with septic tank systems in an urban watershed.. *Ecological Applications* 14: 637-641 \\ 
   32 & Sabo, J. L. *et al*., The role of discharge variation in scaling of drainage area and food chain length in rivers. *Science* 330: 965-967 \\ 
   33 & Itakura, H. *et al*., Anguillid eels as a surrogate species for conservation of freshwater biodiversity in Japan. *Scientific Reports* 10: NA \\ 
   34 & Burdon, F.J. *et al*., Mechanisms of trophic niche compression: Evidence from landscape disturbance. *Journal of Animal Ecology* 89: 730-744 \\ 
   35 & Fraley, K.M. *et al*., Influence of maternally-transferred nitrogen and carbon on stable isotope ratios in juvenile chinook salmon. *North American Journal of Fisheries Management* 40: 175-181 \\ 
   36 & Maitland, B.M. and Rahel, F.J.Aquatic food web expansion and trophic redundancy along the Rocky Mountain–Great Plains ecotone. *Ecology* 104: e4103 \\ 
   37 & Dekar, M.P. *et al*., Shifts in the trophic base of intermittent stream food webs. *Hydrobiologia* 635: 263-277 \\ 
   38 & Doi, H. *et al*., Contribution of chemoautotrophic production to freshwater macroinvertebrates in a headwater stream using multiple stable isotopes. *International Review of Hydrobiology* 91: 501-508 \\ 
   39 & Hoeinghaus, D.J. *et al*., Hydrogeomorphology and river impoundment affect food-chain length of diverse Neotropical food webs. *Oikos* 117: 984-995 \\ 
   40 & Huang, I.Y. *et al*., Food web structure of a subtropical headwater stream. *Marine and Freshwater Research* 58: 596-607 \\ 
   41 & Jepsen, D.B. and Winemiller, K.O.Basin geochemistry and isotopic ratios of fishes and basal production sources in four neotropical rivers. *Ecology of Freshwater Fish* 16: 267-281 \\ 
   42 & Mercado‐Silva, N. *et al*., The effects of impoundment and non-native species on a river food web in Mexico's central plateau. *River Research and Applications* 25: 1090-1108 \\ 
   43 & Reid, D.J. *et al*., Terrestrial detritus supports the food webs in lowland intermittent streams of south-eastern Australia: a stable isotope study. *Freshwater Biology* 53: 2036-2050 \\ 
   44 & Takamura, K., Population structuring by weirs and the effect on trophic position of a freshwater fish Zacco platypus in the middle reaches of Japanese rivers. *Fundamental and Applied Limnology* 174: 307-315 \\ 
   45 & Ulseth, A.J. and Hershey, A.ENatural abundances of stable isotopes trace anthropogenic N and C in an urban stream. *Journal of the North American Benthological Society* 24: 270-289 \\ 
   46 & Winemiller, K.O. *et al*., Stable isotope analysis reveals food web structure and watershed impacts along the fluvial gradient of a Mesoamerican coastal river. *River Research and Applications* 27: 791-803 \\ 
   \hline
\end{tabularx}
\endgroup
\end{table}
